\documentclass[12pt, a4paper]{ctexart}
\usepackage{amsmath, amsthm, amssymb}
\usepackage{listings}
\usepackage[usenames,dvipsnames]{xcolor}
\usepackage{CJK}

\definecolor{mygreen}{rgb}{0,0.6,0}
\definecolor{mygray}{rgb}{0.5,0.5,0.5}
\definecolor{mymauve}{rgb}{0.58,0,0.82}
\lstset{
	%backgroundcolor=\color{lightgray},
	basicstyle = \footnotesize,
	breakatwhitespace = false,
	breaklines = true,
	captionpos = b,
	commentstyle = \color{mygreen}\bfseries,
	extendedchars = false,
	frame =shadowbox,
	framerule=0.5pt,
	keepspaces=true,
	keywordstyle=\color{blue}\bfseries, % keyword style
	language = C++,                     % the language of code
	otherkeywords={string}, 
	numbers=left, 
	numbersep=5pt,
	numberstyle=\tiny\color{mygray},
	rulecolor=\color{black},
	showspaces=false,
	showstringspaces=false, 
	showtabs=false,
	stepnumber=1,
	stringstyle=\color{mymauve},        % string literal style
	tabsize=2,
	title=\lstname
}

\title{hz的板子}
\author{hz}
\date{\today}

\begin{document}
	\maketitle
	\newpage
	\tableofcontents
	\newpage
	\section{高精度及优化技巧}
	\subsection{快读快写}
	\begin{lstlisting}[caption={}]
	#include <bits/stdc++.h>
	using namespace std;

	inline void read(int &n)
	{
		int x = 0, f = 1;
		char ch = getchar();
		while (ch < '0' || ch > '9')
		{
			if (ch == '-')
			f = -1;
			ch = getchar();
		}
		while (ch >= '0' && ch <= '9')
		{
			x = (x << 1) + (x << 3) + (ch ^ 48);
			ch = getchar();
		}
		n = x * f;
	}
	inline void write(int n)
	{
		if (n < 0)
		{
			putchar('-');
			n *= -1;
		}
		if (n > 9)
		write(n / 10);
		putchar(n % 10 + '0');
	}
	\end{lstlisting}
	\subsection{边读入边取模}
	\begin{lstlisting}[caption={}]
		#include <bits/stdc++.h>
		using namespace std;
		
		inline int read(int mod)
		{
			int x = 0;
			bool g = false;
			char c = getchar();
			while (c < '0' || c > '9')
			c = getchar();
			while (c >= '0' && c <= '9')
			{
				x = (x << 3) + (x << 1) + (c ^ '0');
				if (x >= mod)
				x %= mod, g = true;
				c = getchar();
			}
			if (g)
			return (x + mod);
			else
			return x;
		}
	\end{lstlisting}
	\subsection{离散化}
	\begin{lstlisting}[caption={}]
		// arr[i] 为初始数组,下标范围为 [1, n]
		
		for (int i = 1; i <= n; ++i)
			tmp[i] = arr[i];
		std::sort(tmp + 1, tmp + n + 1);                          
		int len = std::unique(tmp + 1, tmp + n + 1) - (tmp + 1);
		for (int i = 1; i <= n; ++i)                              
			arr[i] = std::lower_bound(tmp + 1, tmp + len + 1, arr[i]) - tmp;
	\end{lstlisting}
	或
	\begin{lstlisting}[caption={}]
		// std::vector<int> arr;
		std::vector<int> tmp(arr);  // tmp 是 arr 的一个副本
		std::sort(tmp.begin(), tmp.end());
		tmp.erase(std::unique(tmp.begin(), tmp.end()), tmp.end());
		for (int i = 0; i < n; ++i)
			arr[i] = std::lower_bound(tmp.begin(), tmp.end(), arr[i]) - tmp.begin();
	\end{lstlisting}
	\subsection{高精度}
	\subsubsection{普通整数高精度}
	\begin{lstlisting}[caption={}]
		#include <bits/stdc++.h>
		using namespace std;
		
		struct Number
		{
			string s;
			bool sign = 0;
		};
		int compare(string str1, string str2)
		{
			if (str1.length() > str2.length())
			return 1;
			else if (str1.length() < str2.length())
			return -1;
			else
			return str1.compare(str2);
		}
		string add(string str1, string str2)
		{
			string str;
			int len1 = str1.length();
			int len2 = str2.length();
			if (len1 < len2)
			{
				for (int i = 1; i <= len2 - len1; i++)
				str1 = "0" + str1;
			}
			else
			{
				for (int i = 1; i <= len1 - len2; i++)
				str2 = "0" + str2;
			}
			len1 = str1.length();
			int cf = 0;
			int temp;
			for (int i = len1 - 1; i >= 0; i--)
			{
				temp = str1[i] - '0' + str2[i] - '0' + cf;
				cf = temp / 10;
				temp %= 10;
				str = char(temp + '0') + str;
			}
			if (cf != 0)
			str = char(cf + '0') + str;
			return str;
		}
		string sub(string str1, string str2)
		{
			string str;
			int tmp = str1.length() - str2.length();
			int cf = 0;
			for (int i = str2.length() - 1; i >= 0; i--)
			{
				if (str1[tmp + i] < str2[i] + cf)
				{
					str = char(str1[tmp + i] - str2[i] - cf + '0' + 10) + str;
					cf = 1;
				}
				else
				{
					str = char(str1[tmp + i] - str2[i] - cf + '0') + str;
					cf = 0;
				}
			}
			for (int i = tmp - 1; i >= 0; i--)
			{
				if (str1[i] - cf >= '0')
				{
					str = char(str1[i] - cf) + str;
					cf = 0;
				}
				else
				{
					str = char(str1[i] - cf + 10) + str;
					cf = 1;
				}
			}
			str.erase(0, str.find_first_not_of('0'));
			return str;
		}
		string mul(string str1, string str2)
		{
			string str;
			int len1 = str1.length();
			int len2 = str2.length();
			string tempstr;
			for (int i = len2 - 1; i >= 0; i--)
			{
				tempstr = "";
				int temp = str2[i] - '0';
				int t = 0;
				int cf = 0;
				if (temp != 0)
				{
					for (int j = 1; j <= len2 - 1 - i; j++)
					tempstr += "0";
					for (int j = len1 - 1; j >= 0; j--)
					{
						t = (temp * (str1[j] - '0') + cf) % 10;
						cf = (temp * (str1[j] - '0') + cf) / 10;
						tempstr = char(t + '0') + tempstr;
					}
					if (cf != 0)
					tempstr = char(cf + '0') + tempstr;
				}
				str = add(str, tempstr);
			}
			str.erase(0, str.find_first_not_of('0'));
			return str;
		}
		void div(string str1, string str2, string &quotient, string &residue)
		{
			quotient = residue = "";
			if (str2 == "0")
			{
				quotient = residue = "ERROR";
				return;
			}
			if (str1 == "0")
			{
				quotient = residue = "0";
				return;
			}
			int res = compare(str1, str2);
			if (res < 0)
			{
				quotient = "0";
				residue = str1;
				return;
			}
			else if (res == 0)
			{
				quotient = "1";
				residue = "0";
				return;
			}
			else
			{
				int len1 = str1.length();
				int len2 = str2.length();
				string tempstr;
				tempstr.append(str1, 0, len2 - 1);
				for (int i = len2 - 1; i < len1; i++)
				{
					tempstr = tempstr + str1[i];
					tempstr.erase(0, tempstr.find_first_not_of('0'));
					if (tempstr.empty())
					tempstr = "0";
					for (char ch = '9'; ch >= '0'; ch--)
					{
						string str, tmp;
						str = str + ch;
						tmp = mul(str2, str);
						if (compare(tmp, tempstr) <= 0)
						{
							quotient = quotient + ch;
							tempstr = sub(tempstr, tmp);
							break;
						}
					}
				}
				residue = tempstr;
			}
			quotient.erase(0, quotient.find_first_not_of('0'));
			if (quotient.empty())
			quotient = "0";
		}
	\end{lstlisting}
	
	\subsubsection{小数除法高精}
	
	不会,别急
	
	\subsubsection{fft}
	
	(不会用, 能不用就不用)
	
	\begin{lstlisting}[caption={}]
		#include<stdio.h>
		#include<string.h>
		typedef unsigned char byte;
		typedef unsigned int word;
		typedef unsigned long long ull;
		typedef long long ll;
		const word mod=1004535809,size=21;
		//NTT模数,结果不超过(1<<size) 
		char io[(1<<size)+1];//输入输出 
		word a[1<<size],b[1<<size];//多项式/高精 
		word realid[1<<size],root[1<<size],inverse[1<<size];
		//迭代后系数所在的位置,单位根及其逆元 
		word i,id,floor;
		ll num1,num2;
		char *top;
		//循环变量 
		inline ll pow(ll a,ll b){//快速幂 
			register ll ans=1;
			for(;b;b>>=1){
				if(b&1) (ans*=a)%=mod;
				(a*=a)%=mod;
			}
			return ans;
		}
		inline void loading(){//预处理迭代后位置,单位根及其逆元 
			root[0]=inverse[0]=1;
			num1=pow(3,mod>>size);
			num2=pow(num1,mod-2);
			for(i=1;i<1<<size;i++){
				root[i]=num1*root[i-1]%mod;
				inverse[i]=num2*inverse[i-1]%mod;
				for(id=i,floor=0;floor<size;floor++,id>>=1)
				realid[i]=realid[i]<<1|(id&1);
			}
		}
		inline void read(){
			//直接用fread,然后指针前移读入数据,效率较高,写法较简单 
			//因fread特性,DEV-CPP下不能以数字结尾(不停读入最后一个字符),但linux下可以 
			top=io+(1<<size);
			fread(io+1,1,1<<size,stdin);
			while('0'>*top||*top>'9') top--;
			for(i=0;'0'<=*top&&*top<='9';top--,i++)
			a[realid[i]]=*top-'0';//直接放到迭代后的位置 
			while('0'>*top||*top>'9') top--;	
			for(i=0;'0'<=*top&&*top<='9';top--,i++)
			b[realid[i]]=*top-'0';//b同理 
		} 
		inline void DFT(){ //非迭代版DFT 
			for(floor=0;floor<size;floor++)
			for(i=0;i<1<<size;i+=(1<<(floor+1)))
			for(id=0;id<1<<floor;id++){
				num1=a[i+id];//蝴蝶变换 
				num2=a[i+id+(1<<floor)];
				(num2*=root[id<<size-floor-1])%=mod;
				a[i+id]=(num1+num2)%mod;//放回原位 
				a[i+id+(1<<floor)]=(num1+mod-num2)%mod;
				
				num1=b[i+id];//b同理 
				num2=b[i+id+(1<<floor)];
				(num2*=root[id<<size-floor-1])%=mod;
				b[i+id]=(num1+num2)%mod;
				b[i+id+(1<<floor)]=(num1+mod-num2)%mod;
			}
		}
		inline void IDFT(){
			for(floor=0;floor<21;floor++)//与DFT相同 
			for(i=0;i<0x200000;i+=1<<floor+1)
			for(id=0;id<1<<floor;id++){
				num1=root[i+id];
				num2=root[i+id+(1<<floor)];
				(num2*=inverse[id<<20-floor])%=mod;//乘上单位根逆元 
				root[i+id]=(num1+num2)%mod;
				root[i+id+(1<<floor)]=(num1+mod-num2)%mod;
			}
		}
		inline void write(){//fwrite输出,同输入
			num2=pow(1<<size,mod-2);
			top=io+(1<<size);
			num1=0;
			for(i=0;i<1<<size;i++){
				num1+=num2*root[i]%mod;//最后要乘上n的逆元 
				top--;
				*top=num1%10+'0';
				num1/=10;
			}
			while(*top=='0'&&top!=io+(1<<size)) top++;
			if(top==io+(1<<size)) putchar('0');
			else fwrite(top,1,io+(1<<size)-top,stdout);
		}
		int main(){
			loading();
			read(); 
			DFT();
			for(i=0;i<1<<size;i++)
			root[realid[i]]=(ll)(a[i])*b[i]%mod;//点值相乘 
			IDFT();
			write();
			return 0;
		}
	\end{lstlisting}
	\subsection{随机数}
	\begin{lstlisting}[caption={}]
		template <class T>
		T randint(T l, T r = 0) // 生成随机数建议用<random>里的引擎和分布,而不是rand()模数,那样保证是均匀分布
		{
			static mt19937 eng(time(0));
			if (l > r)
			swap(l, r);
			uniform_int_distribution<T> dis(l, r);
			return dis(eng);
		}
	\end{lstlisting}
	\subsection{手写哈希}
	\begin{lstlisting}[caption={}]
		template<typename key_t, typename type> struct hash_table {
			static const int maxn = 1000010;
			static const int table_size = 11110007;
			int first[table_size], nxt[maxn], sz; //init: memset(first, 0, sizeof(first)), sz = 0
			key_t id[maxn];
			type data[maxn];
			type& operator[] (key_t key) {
				const int h = key % table_size;
				for (int i = first[h]; i; i = nxt[i])
				if (id[i] == key)
				return data[i];
				int pos = ++sz;
				nxt[pos] = first[h];
				first[h] = pos;
				id[pos] = key;
				return data[pos] = type();
			}
			bool count(key_t key) {
				for (int i = first[key % table_size]; i; i = nxt[i])
				if (id[i] == key)
				return true;
				return false;
			}
			type get(key_t key) { //如果key对应的值不存在,则返回type()。
				for (int i = first[key % table_size]; i; i = nxt[i])
				if (id[i] == key)
				return data[i];
				return type();
			}
		};
	\end{lstlisting}
	\subsection{linux对拍}
	\begin{lstlisting}[caption={}]
		#include <algorithm>
		using namespace std;
		int main(){
			int T = 10000;
			int tot = 0;
			while(T--){
				tot++;
				cout << tot << " ";
				system("./rand; ./std; ./tmp");
				if(system("diff std.out tmp.out")){
					cout << "WA" << endl;
					return 0;
				}
				else cout << "AC" << endl;
			}
		}
	\end{lstlisting}
	\begin{lstlisting}[caption={}]
		make data
		make 001
		make 002
		
		((cnt=1))
		
		while true
		do
		./data > in
		./001 < in > 1.out
		./002 < in > 2.out
		if diff 1.out 2.out; then
		printf "# $((cnt++)) Accepted\n";
		else 
		notify-send "Gary"
		break
		fi
		done
	\end{lstlisting}
	\section{简单算法备忘录}
	\subsection{三分答案}
	\begin{lstlisting}[caption={}]
		ll ans = 1e18;
		while (l + 1 < r) // 直到只剩不到三个数为止
		{
			mid = (l + r) / 2;
			ll a1 = run(mid - 1), a2 = run(mid + 1);
			if (a1 > a2)
			ans = min(ans, a1), l = mid;
			else
			ans = min(ans, a2), r = mid;
		}
		ans = min(ans, run(l));
		ans = min(ans, run(r));
		cout << ans << endl;
	\end{lstlisting}
	\subsection{树状数组}
	\begin{lstlisting}[caption={}]
		void add(int k, int x) {
			while (k < N - 100) {
				tree[k] += x;
				k += lowbit(k);
			}
		}
		int sum(int k) {
			int ans = 0;
			while (k) {
				ans += tree[k];
				k -= lowbit(k);
			}
			return ans;
		}
	\end{lstlisting}
	\subsection{线段树}
	\begin{lstlisting}[caption={}]
		struct seg_tree {
			using type = i64;
			#define ls(x) x << 1
			#define rs(x) x << 1 | 1
			#define val(x) tree[x].val
			#define tag(x) tree[x].tag
			
			struct node {
				int ls, rs;
				type val, tag;
			}tree[N << 2];
			void pushdown(int p, int len) {
				if (len == 1) {
					return;
				}
				tag(ls(p)) = (tag(p) + tag(ls(p))) % mod;
				tag(rs(p)) = (tag(p) + tag(rs(p))) % mod;
				val(ls(p)) = (val(ls(p)) + tag(p) * (len - len / 2) % mod) % mod;
				val(rs(p)) = (val(rs(p)) + tag(p) * (len / 2) % mod) % mod;
				tag(p) = 0;
			}
			void pushup(int p) {
				val(p) = (val(ls(p)) + val(rs(p))) % mod;
			}
			void build(vector<int> &a, int l = 1, int r = n, int p = 1) {
				if (l == r) {
					val(p) = a[l - 1];
					return;
				}
				int mid = l + r >> 1;
				build(a, l, mid, ls(p));
				build(a, mid + 1, r, rs(p));
				pushup(p);
			}
			void update(int l, int r, int k, int nl = 1, int nr = n, int p = 1) {
				if (l <= nl && r >= nr) {
					val(p) = (val(p) + k * (nr - nl + 1) % mod) % mod;
					if (nl != nr) {
						tag(p) = (tag(p) + k) % mod;
					}
					return;
				}
				pushdown(p, nr - nl + 1);
				int mid = nl + nr >> 1;
				if (mid >= l) {
					update(l, r, k, nl, mid, ls(p));
				}
				if (mid < r) {
					update(l, r, k, mid + 1, nr, rs(p));
				}
				pushup(p);
			}
			type query(int l, int r, int nl = 1, int nr = n, int p = 1) {
				if (l <= nl && r >= nr) {
					return val(p);
				}
				pushdown(p, nr - nl + 1);
				int mid = nl + nr >> 1;
				type ans = 0;
				if (mid >= l) {
					ans = (ans + query(l, r, nl, mid, ls(p))) % mod;
				}
				if (mid < r) {
					ans = (ans + query(l, r, mid + 1, nr, rs(p))) % mod;
				}
				return ans;
			}
			void debug(int num = 1) {
				for (int i = 1; i <= n; i += num) {
					query(i, min(n, i + num - 1));
				}
				// cout << endl;
			}
		}
	\end{lstlisting}
	\subsection{权值线段树}
	\begin{lstlisting}[caption={}]
		void insert(int v) // 插入
		{
			update(v, 1);
		}
		void remove(int v) // 删除
		{
			update(v, -1);
		}
		int countl(int v)
		{
			return query(L, v - 1);
		}
		int countg(int v)
		{
			return query(v + 1, R);
		}
		int rank(int v) // 求排名
		{
			return countl(v) + 1;
		}
		int kth(int k, int p = 1, int cl = L, int cr = R) // 求指定排名的数
		{
			if (cl == cr)
			return cl;
			int mid = (cl + cr - 1) / 2;
			if (val(ls(p)) >= k)
			return kth(k, ls(p), cl, mid); // 往左搜
			else
			return kth(k - val(ls(p)), rs(p), mid + 1, cr); // 往右搜
		}
		int pre(int v) // 求前驱
		{
			int r = countl(v);
			return kth(r);
		}
		int suc(int v) // 求后继
		{
			int r = val(1) - countg(v) + 1;
			return kth(r);
		}
	\end{lstlisting}
	\subsection{动态开点线段树}
	$MAXV$一般能开多大开多大,例如内存限制$128M$时可以开到八百万左右
	\begin{lstlisting}[caption={}]
		#define ls(x) tree[x].ls
		#define rs(x) tree[x].rs
		#define val(x) tree[x].val
		#define mark(x) tree[x].mark
		const int MAXV = 8e6;
		int L = 1, R = 1e5, cnt = 1;
		struct node
		{
			ll val, mark;
			int ls, rs;
		} tree[MAXV];
		void upd(int &p, int x, int len)
		{
			if (!p) p = ++cnt;
			val(p) += x * len;
			mark(p) += x;
		}
		void push_down(int p, int len)
		{
			if (len <= 1) return;
			upd(ls(p), mark(p), len - len / 2);
			upd(rs(p), mark(p), len / 2);
			mark(p) = 0;
		}
		ll query(int l, int r, int p = 1, int cl = L, int cr = R)
		{
			if (cl >= l && cr <= r) return val(p);
			push_down(p, cr - cl + 1);
			ll mid = (cl + cr - 1) / 2, ans = 0;
			if (mid >= l) ans += query(l, r, ls(p), cl, mid);
			if (mid < r) ans += query(l, r, rs(p), mid + 1, cr);
			return ans;
		}
		void update(int l, int r, int d, int p = 1, int cl = L, int cr = R)
		{
			if (cl >= l && cr <= r) return (void)(val(p) += d * (cr - cl + 1), mark(p) += d);
			push_down(p, cr - cl + 1);
			int mid = (cl + cr - 1) / 2;
			if (mid >= l) update(l, r, d, ls(p), cl, mid);
			if (mid < r) update(l, r, d, rs(p), mid + 1, cr);
			val(p) = val(ls(p)) + val(rs(p));
		}
	\end{lstlisting}
	\section{数据结构}
	\subsection{分块}
	\begin{lstlisting}[caption={}]
		int main() {
			int n, m;
			cin >> n >> m;
			int sq = sqrt(n);
			for (int i = 1; i <= sq; ++i) {
				st[i] = ed[i - 1] + 1;
				ed[i] = n / sq * i;
			}
			ed[sq] = n;
			for (int i = 1; i <= sq; ++i) {
				for (int j = st[i]; j <= ed[i]; ++j) {
					bel[j] = i;
				}
			}
			for (int i = 1; i <= sq; ++i) {
				size[i] = ed[i] - st[i] + 1;
			}
	\end{lstlisting}
	\subsection{树状数组}
	nlgn, 单点修改, 区间查询
	
	\begin{lstlisting}[caption={}]
		int lowbit(int x) {
			return x & (-x);
		}
		void add(int k, int x) {
			while (k < N - 100) {
				tree[k] += x;
				k += lowbit(k);
			}
			return;
		}
		
		int sum(int x) {
			int ans = 0;
			while (x) {
				ans += tree[x];
				x -= lowbit(x);
			}
			return ans;
		}
	\end{lstlisting}
	\subsection{ST表}
	\begin{lstlisting}[caption={}]
		#include <bits/stdc++.h>
		using namespace std;
		
		const int N = 1e5 + 1e3;
		
		int f[N][21];
		
		int main() {
			int n, m;
			cin >> n >> m;
			for (int i = 1; i <= n; ++i) {
				int x;
				scanf ("%d", &x);
				f[i][0] = x;
			}
			for (int i = 1; i <= 20; ++i) {
				for (int j = 1; j + (1 << i) - 1 <= n; ++j) {
					f[j][i] = max(f[j][i - 1], f[j + (1 << (i - 1))][i - 1]);
				}
			}
			while (m--) {
				int l, r;
				scanf ("%d %d", &l, &r);
				int s = __lg(r - l + 1);
				printf("%d\n", max(f[l][s], f[r - (1 << s) + 1][s]));
			}
			return 0;
		}
	\end{lstlisting}
	\subsection{线段树}
	\subsubsection{最大字段和}
	\begin{lstlisting}[caption={}]
		#include <bits/stdc++.h>
		using namespace std;
		#define sum(x) tree[x].sum
		#define maxl(x) tree[x].maxl
		#define maxr(x) tree[x].maxr
		#define ans(x) tree[x].ans
		#define mark(x) tree[x].mark
		
		const int N = 5e5 + 100;
		
		typedef struct node {
			int sum, maxl, maxr, ans;
			int mark;
			node() {
				sum = maxl = maxr = ans = 0;
				mark = 1e9;
			}
			node(int x) {
				sum = x;
				maxl = x;
				maxr = x;
				ans = x;
				mark = 1e9;
			}
		}T;
		
		T init(int x) {
			T node(x);
			return node;
		}
		T tree[N << 2];
		int a[N];
		int n;
		void push_down(int p) {
			if (mark(p) != 1e9) {
				tree[p << 1] = tree[p << 1 | 1] = init(mark(p));
				mark(p << 1) = mark(p << 1 | 1) = mark(p);
			}
		}
		void push_up(int p) {
			sum(p) = sum(p << 1) + sum(p << 1 | 1);
			maxl(p) = max(maxl(p << 1), sum(p << 1) + maxl(p << 1 | 1));
			maxr(p) = max(maxr(p << 1 | 1), sum(p << 1 | 1) + maxr(p << 1));
			ans(p) = max(maxr(p << 1) + maxl(p << 1 | 1), max(ans(p << 1), ans(p << 1 | 1)));
		}
		T push_up(T A, T B) {
			T nn;
			nn.sum = A.sum + B.sum;
			nn.maxl = max(A.maxl, A.sum + B.maxl);
			nn.maxr = max(B.maxr, B.sum + A.maxr);
			nn.ans = max(A.maxr + B.maxl, max(A.ans, B.ans));
			return nn;
		}
		void build(int p = 1, int l = 1, int r = n) {
			if (l == r) {
				tree[p] = init(a[l]);
				return;
			}
			int mid = l + r >> 1;
			build(p << 1, l, mid);
			build(p << 1 | 1, mid + 1, r);
			push_up(p);
		}
		void update(int l, int r, int k, int p = 1, int cl = 1, int cr = n) {
			if (l > cr || r < cl) {
				return;
			}
			if (l <= cl && r >= cr) {
				tree[p] = init(k);
				return;
			}
			int mid = cl + cr >> 1;
			push_down(p);
			update(l, r, k, p << 1, cl, mid);
			update(l, r, k, p << 1 | 1, mid + 1, cr);
			push_up(p);
		}
		T query(int l, int r, int p = 1, int cl = 1, int cr = n) {
			if (l <= cl && r >= cr) {
				return tree[p];
			}
			int mid = cl + cr >> 1;
			push_down(p);
			if (l > mid) {
				return query(l, r, p << 1 | 1, mid + 1, cr);
			}
			if (r < mid + 1) {
				return query(l, r, p << 1, cl, mid);
			}
			return push_up(query(l, r, p << 1, cl, mid), query(l, r, p << 1 | 1, mid + 1, cr));
		}
		int main() {
			int m;
			cin >> n >> m;
			for (int i = 1; i <= n; ++i) {
				cin >> a[i];
			}
			build();
			while (m--) {
				int op;
				cin >> op;
				if (op == 1) {
					int a, b;
					cin >> a >> b;
					if (a > b) {
						swap(a, b);
					}
					cout << query(a, b).ans << endl;
				} else {
					int p, s;
					cin >> p >> s;
					update(p, p, s);
				}
			}
			system("pause");
		}
	\end{lstlisting}
	\subsection{可持久化}
	\subsubsection{可持久化数组(单点修改单点查询)}
	\begin{lstlisting}[caption={}]
		#include <bits/stdc++.h>
		using namespace std;
		
		const int N = 3e7 + 1e2;
		const int M = 1e6 + 1e2;
		
		#define ls(x) tree[x].ls
		#define rs(x) tree[x].rs
		#define sum(x) tree[x].sum
		
		typedef struct node {
			int ls, rs;
			int sum;
		}T;
		
		T tree[N];
		int a[M];
		int root[M];
		int cnt;
		
		int n;
		
		void push_up(int p) {
			sum(p) = sum(ls(p)) + sum(rs(p));
		}
		void build(int p = cnt, int l = 1, int r = n) {
			if (l == r) {
				sum(p) = a[l];
				return;
			}
			int mid = l + r >> 1;
			ls(p) = ++cnt;
			rs(p) = ++cnt;
			build(ls(p), l, mid);
			build(rs(p), mid + 1, r);
		}
		void update(int q, int k, int root, int p, int cl = 1, int cr = n) {
			//cout << q << ' ' << k << ' ' << root << ' ' << p << ' ' << cl << ' ' << cr << endl;
			if (cl == cr) {
				sum(p) = k;
				return;
			}
			int mid = cl + cr >> 1;
			if (q <= mid) {
				ls(p) = ++cnt;
				rs(p) = rs(root);
				update(q, k, ls(root), ls(p), cl, mid);
			} else {
				ls(p) = ls(root);
				rs(p) = ++cnt;
				update(q, k, rs(root), rs(p), mid + 1, cr);
			}
			push_up(p);
		}
		int query(int q, int p, int cl = 1, int cr = n) {
			if (cl == cr) {
				return sum(p);
			}
			int mid = cl + cr >> 1;
			if (q <= mid) {
				return query(q, ls(p), cl, mid);
			} else {
				return query(q, rs(p), mid + 1, cr);
			}
		}
		int main() {
			ios::sync_with_stdio(0);
			int m;
			cin >> n >> m;
			for (int i = 1; i <= n; ++i) {
				cin >> a[i];
			}
			root[0] = ++cnt;
			build();
			for (int i = 1; i <= m; ++i) {
				int op, a, b;
				cin >> a >> op >> b;
				if (op == 1) {
					root[i] = ++cnt;
					int k;
					cin >> k;
					update(b, k, root[a], root[i]);
				} else {
					root[i] = root[a];
					cout << query(b, root[a]) << endl;
				}
			}
			system("pause");
		}
	\end{lstlisting}
	\subsubsection{主席树(区间第k小)}
	\begin{lstlisting}[caption={}]
		#include <bits/stdc++.h>
		using namespace std;
		
		#define int long long
		#define val(x) tree[x].val
		#define ls(x) tree[x].ls
		#define rs(x) tree[x].rs
		
		const int N = 1e6;
		
		typedef struct node {
			int val;
			int ls, rs;
		}T;
		T tree[N << 5];
		
		int roots[N << 5], cnt = 1;
		int a[N + 1000], b[N + 1000], c[N + 1000];
		
		int dis(int n) {
			memcpy(b, a, sizeof a);
			sort(b + 1, b + n + 1);
			int x = unique(b + 1, b + n + 1) - b - 1;
			for (int i = 1; i <= n; ++i) {
				c[i] = lower_bound(b + 1, b + x + 1, a[i]) - b;
			}
			return x;
		}
		
		void build(int p, int l, int r) {
			val(p) = 0;
			if (l == r) {
				return;
			}
			ls(p) = ++cnt, rs(p) = ++cnt;
			int mid = (l + r) >> 1;
			build(ls(p), l, mid);
			build(rs(p), mid + 1, r);
		}
		void update(int x, int p, int q, int l, int r) {
			val(q) = val(p) + 1;
			if (l == r) {
				return;
			}
			ls(q) = ls(p), rs(q) = rs(p);
			int mid = (l + r) >> 1;
			if (x <= mid) {
				ls(q) = ++cnt;
				update(x, ls(p), ls(q), l, mid);
			} else {
				rs(q) = ++cnt;
				update(x, rs(p), rs(q), mid + 1, r);
			}
		}
		
		int query(int k, int p, int q, int l, int r) {
			if (l == r) {
				return b[l];
			}
			int mid = (l + r) >> 1;
			//cout << l << " " << r << endl;
			if (val(ls(q)) - val(ls(p)) >= k) {
				return query(k, ls(p), ls(q), l, mid);
			} else {
				return query(k - val(ls(q)) + val(ls(p)), rs(p), rs(q), mid + 1, r);
			}
		}
		main() {
			ios::sync_with_stdio(0);
			int n, m;
			cin >> n >> m;
			for (int i = 1; i <= n; ++i) {
				cin >> a[i];
			}
			int x = dis(n);
			build(1, 1, x);
			roots[0] = 1;
			for (int i = 1; i <= n; ++i) {
				roots[i] = ++cnt;
				update(c[i], roots[i - 1], roots[i], 1, x);
			}
			while (m--) {
				int l, r, k;
				cin >> l >> r >> k;
				cout << query(k, roots[l - 1], roots[r], 1, x) << endl;
			}
			return 0;
		}
	\end{lstlisting}
	\subsection{线段树合并}
	\begin{lstlisting}[caption={}]
		namespace segtree {
			typedef struct node {
				int ls, rs;
				int val, id;
			}T;
			int n = 1e5, cnt;
			T tree[M];
			int root[N];
			void pushup(int p) {
				val(p) = std::max(val(ls(p)), val(rs(p)));
				if (val(ls(p)) >= val(rs(p))) {
					id(p) = id(ls(p));
				} else {
					id(p) = id(rs(p));
				}
			}
			void add(int pos, int d, int &p, int l = 1, int r = n) {
				//std::cout << pos << ' ' << d << ' ' << p << ' ' << l << ' ' << r << std::endl;
				if (!p) {
					p = ++cnt;
				}
				if (l == r) {
					id(p) = l;
					val(p) += d;
					return;
				}
				int mid = l + r >> 1;
				if (pos <= mid) {
					add(pos, d, ls(p), l, mid);
				} else {
					add(pos, d, rs(p), mid + 1, r);
				}
				pushup(p);
				return;
			}
			int merge(int p, int q, int l = 1, int r = n) {
				if (!p || !q) {
					return p + q;
				}
				if (l == r) {
					val(p) += val(q);
					return p;
				}
				int mid = l + r >> 1;
				ls(p) = merge(ls(p), ls(q), l, mid);
				rs(p) = merge(rs(p), rs(q), mid + 1, r);
				pushup(p);
				return p;
			}  
		}
	\end{lstlisting}
	\subsection{带权并查集}
	\begin{lstlisting}[caption={}]
		int f[N];
		int d[N];
		void init(int n) {
			for (int i = 0; i <= n; ++i) {
				f[i] = i;
			}
		}
		int find(int x) {
			if (f[x] == x) {
				return x;
			}
			int t = f[x];
			f[x] = find(f[x]);
			d[x] += d[t];
			return f[x];
		}
		void merge(int x, int y, int v) {
			int fx = find(x), fy = find(y);
			if (fx == fy) {
				return;
			}
			f[fx] = fy;
			d[fx] = v + d[y] - d[x];
		}
	\end{lstlisting}
	\subsection{可撤销并查集}
	并查集按大小合并
	\begin{lstlisting}[caption={}]
		class DSU {
			private:
			using p = pair<int, int>;
			const static int N = 2e6;
			int f[N + 100], sz[N + 100];
			vector<p> stk;
			public:
			DSU() {
				for (int i = 0; i <= N; ++i) {
					f[i] = i;
					sz[i] = 1;
				}
			}
			int find(int x) {
				if (f[x] == x) {
					return x;
				}
				return find(f[x]);
			}
			bool merge(int x, int y) {
				int fx = find(x), fy = find(y);
				if (fx == fy) {
					stk.push_back({-1, -1});
					return 0;
				}
				if (sz[fx] > sz[fy]) {
					swap(fx, fy);
				}
				f[fx] = fy;
				sz[fy] += sz[fx];
				stk.push_back({fx, fy});
				return 1;
			}
			bool undo() {
				if (!stk.empty()) {
					auto [x, y] = stk.back();
					stk.pop_back();
					if (x == -1 && y == -1) {
						return 0;
					}
					f[x] = x;
					sz[y] -= sz[x];
					return 1;
				}
				return 0;
			}
		}dsu;
	\end{lstlisting}
	\subsection{莫队}
	$O(1)$时间扩展序列
	\begin{lstlisting}[caption={}]
		#include <bits/stdc++.h>
		using namespace std;
		
		const int MAXN = 5e4 + 100, MAXQ = 5e4 + 100, MAXM = 5e4 + 100;
		int sq;
		struct query // 把询问以结构体方式保存
		{
			int l, r, k, id;
			bool operator<(const query &o) const // 重载<运算符,奇偶化排序
			{
				// 这里只需要知道每个元素归属哪个块,而块的大小都是sqrt(n),所以可以直接用l/sq
				if (l / sq != o.l / sq)
				return l < o.l;
				if (l / sq & 1)
				return r < o.r;
				return r > o.r;
			}
		} Q[MAXQ];
		int A[MAXN], ans[MAXQ], Cnt[MAXM], l = 1, r = 0;
		inline void add(int p){}
		inline void del(int p){}
		void solve(int i) {
			while (l > Q[i].l)
			add(--l);
			while (r < Q[i].r)
			add(++r);
			while (l < Q[i].l)
			del(l++);
			while (r > Q[i].r)
			del(r--);
		}
		int main()
		{
			int n, q;
			cin >> n >> q;
			sq = sqrt(n);
			for (int i = 1; i <= n; ++i) {
				cin >> A[i];
			}
			for (int i = 0; i < q; ++i) {
				cin >> Q[i].l >> Q[i].r >> Q[i].k;
				Q[i].id = i; // 把询问离线下来
			}
			sort(Q, Q + q);                                    // 排序
			for (int i = 0; i < q; ++i)
			{
				solve(i);
				//存储答案
				//ans[Q[i].id] = ?
			}
			for (int i = 0; i < q; ++i)
				printf("%d\n", ans[i]); // 按编号顺序输出
			system("pause");
		}
	\end{lstlisting}
	\subsection{珂朵莉树}
	\begin{lstlisting}[caption={}]
		struct node
		{
			ll l, r;
			mutable ll v;
			node(ll l, ll r, ll v) : l(l), r(r), v(v) {}
			bool operator<(const node &o) const { return l < o.l; }
		};
		set<node> tree;
		auto split(ll pos)
		{
			auto it = tree.lower_bound(node(pos, 0, 0));
			if (it != tree.end() && it->l == pos)
			return it;
			it--;
			ll l = it->l, r = it->r, v = it->v;
			tree.erase(it);
			tree.insert(node(l, pos - 1, v));
			return tree.insert(node(pos, r, v)).first;
		}
		void assign(ll l, ll r, ll v)
		{
			auto end = split(r + 1), begin = split(l);
			tree.erase(begin, end);
			tree.insert(node(l, r, v));
		}
	\end{lstlisting}
	\subsection{平衡树(fhq-Treap)}
	\begin{lstlisting}[caption={}]
		struct node {
			int ls, rs, key, sz, val;
		}fhq[N];
		int cnt;
		int create(int k) {
			fhq[++cnt] = {0, 0, rand(), 1, k};
			return cnt;
		}
		int root;
		void pushup(int u) {
			sz(u) = sz(ls(u)) + sz(rs(u)) + 1;
		}
		void split(int u, int v, int &x, int &y) {
			if (!u) {
				x = y = 0;
				return;
			}
			if (val(u) > v) {
				y = u;
				split(ls(u), v, x, ls(u));
			} else {
				x = u;
				split(rs(u), v, rs(u), y);
			}
			pushup(u);
		}
		int merge(int x, int y) {
			if (!x || !y) {
				return x + y;
			}
			if (key(x) < key(y)) {
				ls(y) = merge(x, ls(y));
				pushup(y);
				return y;
			} else {
				rs(x) = merge(rs(x), y);
				pushup(x);
				return x;
			}
		}
		void insert(int k) {
			int x, y;
			create(k);
			split(root, k, x, y);
			root = merge(x, merge(cnt, y));
		}
		void del(int k) {
			int x, y, z;
			split(root, k, x, y);
			split(x, k - 1, x, z);
			z = merge(ls(z), rs(z));
			root = merge(merge(x, z), y);
		}
		int kth(int k, int u = root) {
			if (sz(ls(u)) + 1 > k) {
				return kth(k, ls(u));
			} else if (sz(ls(u)) + 1 == k) {
				return u;
			} else {
				return kth(k - sz(ls(u)) - 1, rs(u));
			}
		}
		int find_rank(int x) {
			int x1, y1;
			split(root, x - 1, x1, y1);
			int ans = sz(x1) + 1;
			root = merge(x1, y1);
			return ans;
		}
		int find_pre(int x) {
			int x1, y1;
			split(root, x - 1, x1, y1);
			int ans = x1;
			while (rs(ans)) {
				ans = rs(ans);
			}
			root = merge(x1, y1);
			return ans;
		}
		int find_next(int x) {
			int x1, y1;
			split(root, x, x1, y1);
			int ans = y1;
			while (ls(ans)) {
				ans = ls(ans);
			}
			root = merge(x1, y1);
			return ans;
		}
	\end{lstlisting}
	\newpage
	\section{树上问题}
	\subsection{点分治}
	\begin{lstlisting}[caption={}]
		#include <bits/stdc++.h>
		using namespace std;
		#define int long long
		
		const int N = 5e4 + 1e2;
		
		vector<int> g[N];
		int n, k;
		int ans;
		
		namespace CenDec {
			int sz[N];
			int del[N];
			int nn;
			void dfs(int p, int fa = 0) {
				sz[p] = 1;
				int mss = 0;
				for (int x : g[p]) {
					if (!del[x] && x != fa) {
						dfs(x, p);
						if (nn) {
							return;
						}
						sz[p] += sz[x];
						mss = max(mss, sz[x]);
					}
				}
				mss = max(n - sz[p], mss);
				if (mss <= n / 2) {
					nn = p;
					sz[fa] = n - sz[p];
					return;
				}
			}
			int a[N];
			map<int, int> mp;
			void dfs2(int p, int fa, int len) {
				if (len > k) {
					return;
				}
				ans += a[k - len] + (len == k);
				//cout << ans << endl;
				mp[len]++;
				for (int x : g[p]) {
					if (!del[x] && x != fa) {
						dfs2(x, p, len + 1);
					}
				}
			}
			void run(int p) {
				for (int x : g[p]) {
					if (!del[x]) {
						dfs2(x, p, 1);
						for (auto p : mp) {
							a[p.first] += p.second;
						}
						mp.clear();
					}
				}
				del[p] = 1;
				for (int i = 0; i <= N - 100; ++i) {
					a[i] = 0;
				}
				for (int x : g[p]) {
					if (!del[x]) {
						n = sz[x];
						nn = 0;
						dfs(x);
						run(nn);
					}
				}
			}
		}
		main() {
			system("pause");
		}
	\end{lstlisting}
	\subsection{树链剖分}
	\begin{lstlisting}[caption={}]
		#include <bits/stdc++.h>
		using namespace std;
		
		#define int long long
		const int N = 5e5 + 1e2;
		
		vector<int> g[N];
		int dep[N], fa[N], sz[N], hson[N];
		int top[N], dfn[N], mxdfn[N];
		int cnt;
		
		int tree[N << 2], mark[N << 2];
		
		int a[N], b[N];
		int n;
		int mod = LONG_LONG_MAX;
		
		void dfs1(int p, int d = 1)
		{
			dep[p] = d;
			int size = 1, ma = 0;
			for (int u : g[p])
			{
				if (dep[u])
				{
					continue;
				}
				dfs1(u, d + 1);
				fa[u] = p;
				size += sz[u];
				if (sz[u] > ma)
				{
					hson[p] = u;
					ma = sz[u];
				}
			}
			sz[p] = size;
		}
		void dfs2(int p)
		{
			dfn[p] = ++cnt;
			if (hson[p])
			{
				top[hson[p]] = top[p];
				dfs2(hson[p]);
			}
			for (int u : g[p])
			{
				if (top[u])
				{
					continue;
				}
				top[u] = u;
				dfs2(u);
			}
			mxdfn[p] = cnt;
		}
		void push_down(int p, int len)
		{
			mark[p << 1] += mark[p];
			mark[p << 1 | 1] += mark[p];
			tree[p << 1] += mark[p] * (len - len / 2) % mod;
			tree[p << 1 | 1] += mark[p] * (len / 2) % mod;
			mark[p << 1] %= mod;
			mark[p << 1 | 1] %= mod;
			tree[p << 1] %= mod;
			tree[p << 1 | 1] %= mod;
			mark[p] = 0;
		}
		void build(int p = 1, int l = 1, int r = n)
		{
			if (l == r)
			{
				tree[p] = a[l];
				return;
			}
			int mid = (l + r) >> 1;
			build(p << 1, l, mid);
			build(p << 1 | 1, mid + 1, r);
			tree[p] = (tree[p << 1] + tree[p << 1 | 1]) % mod;
		}
		void update(int l, int r, int k, int p = 1, int rl = 1, int rr = n)
		{
			if (l > rr || r < rl)
			{
				return;
			}
			if (l <= rl && r >= rr)
			{
				tree[p] += k * (rr - rl + 1) % mod;
				tree[p] %= mod;
				if (rl != rr)
				{
					mark[p] += k;
					mark[p] %= mod;
				}
				return;
			}
			push_down(p, rr - rl + 1);
			int mid = (rl + rr) >> 1;
			update(l, r, k, p << 1, rl, mid);
			update(l, r, k, p << 1 | 1, mid + 1, rr);
			tree[p] = tree[p << 1] + tree[p << 1 | 1];
			tree[p] %= mod;
		}
		int query(int l, int r, int p = 1, int rl = 1, int rr = n)
		{
			if (l > rr || r < rl)
			{
				return 0;
			}
			if (l <= rl && r >= rr)
			{
				return tree[p];
			}
			int mid = (rl + rr) >> 1;
			push_down(p, rr - rl + 1);
			return (query(l, r, p << 1, rl, mid) + query(l, r, p << 1 | 1, mid + 1, rr)) % mod;
		}
		void update_path(int x, int y, int k)
		{
			while (top[x] != top[y])
			{
				if (dep[top[x]] > dep[top[y]])
				{
					update(dfn[top[x]], dfn[x], k);
					x = fa[top[x]];
				}
				else
				{
					update(dfn[top[y]], dfn[y], k);
					y = fa[top[y]];
				}
			}
			if (dep[x] > dep[y])
			{
				update(dfn[y], dfn[x], k);
			}
			else
			{
				update(dfn[x], dfn[y], k);
			}
		}
		int query_path(int x, int y)
		{
			int ans = 0;
			while (top[x] != top[y])
			{
				if (dep[top[x]] > dep[top[y]])
				{
					ans += query(dfn[top[x]], dfn[x]);
					x = fa[top[x]];
				}
				else
				{
					ans += query(dfn[top[y]], dfn[y]);
					y = fa[top[y]];
				}
				ans %= mod;
			}
			if (dep[x] > dep[y])
			{
				ans += query(dfn[y], dfn[x]);
			}
			else
			{
				ans += query(dfn[x], dfn[y]);
			}
			return ans % mod;
		}
		void update_subtree(int x, int k)
		{
			update(dfn[x], mxdfn[x], k);
		}
		int query_subtree(int x)
		{
			return query(dfn[x], mxdfn[x]);
		}
	\end{lstlisting}
	\subsection{长链剖分}
	\begin{lstlisting}[caption={}]
		int root;
		int hson[N], fa[21][N], len[N], dfn[N], mdfn[N], top[N], dep[N];
		int cnt;
		void dfs1(int u = root, int f = -1)
		{
			if (f != -1)
			{
				fa[0][u] = f;
				dep[u] = dep[f] + 1;
			}
			len[u] = 1;
			for (int x : g[u])
			{
				if (x == f)
				{
					continue;
				}
				dfs1(x, u);
				if (len[u] < len[x] + 1)
				{
					hson[u] = x;
					len[u] = len[x] + 1;
				}
			}
		}
		void dfs2(int u = root, int f = -1)
		{
			dfn[u] = ++cnt;
			if (f == -1)
			{
				top[u] = u;
			}
			for (int x : g[u])
			{
				if (x == f)
				{
					continue;
				}
				if (x == hson[u])
				{
					top[x] = top[u];
				}
				else
				{
					top[x] = x;
				}
				dfs2(x, u);
			}
		}
		void po()
		{
			dfs1();
			dfs2();
		}
	\end{lstlisting}
	\subsection{求树上k级祖先}
	\begin{lstlisting}[caption={}]
		int root;
		int hson[N], fa[21][N], len[N], dfn[N], mdfn[N], top[N], dep[N];
		int cnt;
		void dfs1(int u = root, int f = -1)
		{
			if (f != -1)
			{
				fa[0][u] = f;
				dep[u] = dep[f] + 1;
			}
			int maxn = 0;
			len[u] = 1;
			for (int x : g[u])
			{
				if (x == f)
				{
					continue;
				}
				dfs1(x, u);
				if (len[u] < len[x] + 1)
				{
					hson[u] = x;
					len[u] = len[x] + 1;
				}
			}
			// len[u]++;
		}
		void dfs2(int u = root, int f = -1)
		{
			dfn[u] = ++cnt;
			if (f == -1)
			{
				top[u] = u;
			}
			for (int x : g[u])
			{
				if (x == f)
				{
					continue;
				}
				if (x == hson[u])
				{
					top[x] = top[u];
				}
				else
				{
					top[x] = x;
				}
				dfs2(x, u);
			}
		}
		void po()
		{
			dfs1();
			dfs2();
		}
		int n;
		vector<int> fro[N];
		vector<int> bac[N];
		void pre()
		{
			po();
			for (int i = 1; i <= __lg(n); ++i)
			{
				for (int j = 1; j <= n; ++j)
				{
					fa[i][j] = fa[i - 1][fa[i - 1][j]];
				}
			}
			for (int i = 1; i <= n; ++i)
			{
				if (top[i] == i)
				{
					int p = i;
					for (int j = 1; j <= len[i]; ++j)
					{
						fro[i].push_back(p);
						if (fa[0][p])
						{
							p = fa[0][p];
						}
					}
					p = i;
					for (int j = 1; j <= len[i] + 1; ++j)
					{
						bac[i].push_back(p);
						if (hson[p])
						{
							p = hson[p];
						}
					}
				}
			}
		}
		int query(int x, int k)
		{
			if (k == 0)
			{
				return x;
			}
			int xx = __lg(k);
			int q = fa[xx][x];
			int tt = top[q];
			int d = k - (1 << xx);
			if (dep[q] - dep[tt] >= d)
			{
				return bac[tt][dep[q] - dep[tt] - d];
			}
			else
			{
				return fro[tt][d - dep[q] + dep[tt]];
			}
		}
	\end{lstlisting}
	\subsection{树上启发式合并}
	\begin{lstlisting}[caption={}]
		#include <bits/stdc++.h>
		using namespace std;
		
		const int N = 1e5 + 1e2;
		using i64 = long long;
		vector<int> g[N];
		int hson[N], sz[N];
		int c[N];
		void dfs1(int u = 1, int f = -1) {
			int maxn = 0;
			for (int x : g[u]) {
				if (x == f) {
					continue;
				}
				dfs1(x, u);
				sz[u] += sz[x];
				if (maxn < sz[x]) {
					maxn = sz[x];
					hson[u] = x;
				}
			}
			sz[u]++;
		}
		
		i64 cnt[N], ans[N], sum, maxn;
		void add(int u) {}
		void del(int u) {}
		void subtree(bool flag, int u, int f = -1) {
			if (flag) {
				add(u);
			} else {
				del(u);
			}
			for (int x : g[u]) {
				if (x == f) {
					continue;
				}
				subtree(flag, x, u);
			}
		}
		void dfs(int u = 1, int f = -1, bool flag = 1) {
			for (int x : g[u]) {
				if (x == f || x == hson[u]) {
					continue;
				}
				dfs(x, u, 0);
			}
			if (hson[u]) {
				dfs(hson[u], u, 1);
			}
			for (int x : g[u]) {
				if (x == f || x == hson[u]) {
					continue;
				}
				subtree(1, x, u);
			}
			add(u);
			//subtree(1, u, f);
			ans[u] = sum;
			//sum = 0, maxn = 0;
			if (!flag) {
				subtree(0, u, f);
				sum = maxn = 0;
			}
		}
	\end{lstlisting}
	\subsection{lca}
	\subsubsection{倍增求lca}
	\begin{lstlisting}[caption={}]
		#include <bits/stdc++.h>
		using namespace std;
		
		const int N = 1e5 + 1e2;
		
		vector<int> g[N];
		
		bool vis[N];
		int fa[N][22], depth[N];
		
		void dfs(int x, int dd)
		{
			vis[x] = 1;
			depth[x] = dd;
			for (int i = 1; i <= __lg(depth[x]); ++i)
			{
				fa[x][i] = fa[fa[x][i - 1]][i - 1];
			}
			for (int u : g[x])
			{
				if (!vis[u])
				{
					fa[u][0] = x;
					dfs(u, dd + 1);
				}
			}
		}
		
		int lca(int x, int y)
		{
			if (depth[x] < depth[y])
			{
				swap(x, y);
			}
			while (depth[y] < depth[x])
			{
				x = fa[x][__lg(depth[x] - depth[y])];
			}
			if (x == y)
			{
				return x;
			}
			for (int i = __lg(depth[x]); i >= 0; --i)
			{
				if (fa[x][i] != fa[y][i])
				{
					x = fa[x][i];
					y = fa[y][i];
				}
			}
			return fa[x][0];
		}
	\end{lstlisting}
	\subsubsection{树剖求lca}
	\begin{lstlisting}[caption={}]
		#include <bits/stdc++.h>
		using namespace std;
		
		const int N = 5e5 + 1e2;
		
		vector<int> g[N];
		int dep[N], fa[N], sz[N], hson[N];
		int top[N], dfn[N], mxdfn[N];
		int cnt;
		
		void dfs1(int p, int d = 1)
		{
			dep[p] = d;
			int size = 1, ma = 0;
			for (int u : g[p])
			{
				if (dep[u])
				{
					continue;
				}
				dfs1(u, d + 1);
				fa[u] = p;
				size += sz[u];
				if (sz[u] > ma)
				{
					hson[p] = u;
					ma = sz[u];
				}
			}
			sz[p] = size;
		}
		void dfs2(int p)
		{
			dfn[p] = ++cnt;
			if (hson[p])
			{
				top[hson[p]] = top[p];
				dfs2(hson[p]);
			}
			for (int u : g[p])
			{
				if (top[u])
				{
					continue;
				}
				top[u] = u;
				dfs2(u);
			}
			mxdfn[p] = cnt;
		}
		int lca(int x, int y)
		{
			while (top[x] != top[y])
			{
				if (dep[top[x]] > dep[top[y]])
				{
					x = fa[top[x]];
				}
				else
				{
					y = fa[top[y]];
				}
			}
			return dep[x] > dep[y] ? y : x;
		}
	\end{lstlisting}
	\section{图论}
	\subsection{最短路}
	\subsubsection{dijkstra}
	\begin{lstlisting}[caption={}]
		#include <bits/stdc++.h>
		using namespace std;
		
		const int N = 2e5 + 1e2;
		typedef pair<int, int> p;
		
		vector<p> g[N];
		int dis[N];
		priority_queue<p, vector<p>, greater<p>> pq;
		
		void dijkstra(int x)
		{
			memset(dis, 127, sizeof dis);
			pq.push({0, x});
			dis[x] = 0;
			while (!pq.empty())
			{
				int d = pq.top().first, u = pq.top().second;
				pq.pop();
				if (d > dis[u])
				{
					continue;
				}
				for (auto p : g[u])
				{
					int v = p.first, w = p.second;
					if (w + dis[u] < dis[v])
					{
						dis[v] = dis[u] + w;
						pq.push({dis[v], v});
					}
				}
			}
		}
	\end{lstlisting}
	\subsubsection{spfa}
	\begin{lstlisting}[caption={}]
		#include <bits/stdc++.h>
		using namespace std;
		
		const int N = 2e5 + 1e2;
		typedef pair<int, int> p;
		
		vector<p> g[N];
		int dis[N];
		queue<p> q;
		
		void spfa(int x)
		{
			memset(dis, 127, sizeof dis);
			q.push({0, x});
			dis[x] = 0;
			while (!q.empty())
			{
				int d = q.front().first, u = q.front().second;
				q.pop();
				if (d > dis[u])
				{
					continue;
				}
				for (auto p : g[u])
				{
					int v = p.first, w = p.second;
					if (w + dis[u] < dis[v])
					{
						dis[v] = dis[u] + w;
						q.push({dis[v], v});
					}
				}
			}
		}
	\end{lstlisting}
	
	\newpage
	\section{数学}
	\subsection{线性筛}
	\begin{lstlisting}[caption={}]
		#include <bits/stdc++.h>
		using namespace std;
		
		//#define int long long
		const int N = 1e8 + 1e2;
		
		bool prime[N];
		vector<int> primeset;
		
		void pre(int n) {
			prime[1] = 1;
			for (int i = 2; i <= n; ++i) {
				if (!prime[i]) {
					primeset.push_back(i);
				}
				for (int u : primeset) {
					if (i * u > n) {
						break;
					}
					prime[i * u] = 1;
					if (i % u == 0) {
						break;
					}
				}
			}
		}
	\end{lstlisting}
	\subsection{数论分块}
	求$\sum_{i = 1}^{n}f(i)\lfloor \frac{n}{i} \rfloor$
	\begin{lstlisting}[caption={}]
		int l = 1, r = 0;
		while (l <= n) {
			r = n / (n / l);
			if (r > n) {
				r = n;
			}
			ans += (s(r) - s(l - 1)) * (n / l);
			l = r + 1;
		}
	\end{lstlisting}
	\subsection{欧拉函数}
	\subsubsection{线性求欧拉函数}
	\begin{lstlisting}[caption={}]
		#include <bits/stdc++.h>
		using namespace std;
		
		const int N = 1e7 + 1e2;
		bool prime[N];
		vector<int> primeset;
		int phi[N];
		
		void pre(int n) {
			prime[1] = 1, phi[1] = 1;
			for (int i = 2; i <= n; ++i) {
				if (!prime[i]) {
					primeset.push_back(i);
					phi[i] = i - 1;
				}
				for (int u : primeset) {
					if (i * u > n) {
						break;
					}
					prime[i * u] = 1;
					if (i % u) {
						phi[i * u] = phi[i] * phi[u];
					} else {
						phi[i * u] = phi[i] * u;
						break;
					}
				}
			}
		}
	\end{lstlisting}
	\subsubsection{单个数的欧拉函数}
	\begin{lstlisting}[caption={}]
		int phi(int n)
		{
			int ans = n;
			for (int i = 2; i * i <= n; i++)
			{
				if (n % i == 0)
				{
					ans = ans / i * (i - 1);
					while (n % i == 0)
					n /= i;
				}
			}
			if (n > 1)
			ans = ans / n * (n - 1);
			return ans % mod;
		}
	\end{lstlisting}
	\subsection{扩展欧拉定理}
	求$a^b \bmod p$
	\begin{lstlisting}[caption={}]
		#include <bits/stdc++.h>
		using namespace std;
		#define int long long
		
		const int N = 1e8 + 1e2;
		
		int qpow(int n, int k, int mod)
		{
			int ans = 1;
			while (k)
			{
				if (k % 2)
				{
					ans = (ans * n) % mod;
				}
				k /= 2;
				n = (n * n) % mod;
			}
			return ans;
		}
		int phi(int n)
		{
			int ans = n;
			for (int i = 2; i * i <= n; ++i)
			{
				if (n % i == 0)
				{
					ans = ans / i * (i - 1);
				}
				while (n % i == 0)
				{
					n /= i;
				}
			}
			if (n > 1)
			{
				ans = ans / n * (n - 1);
			}
			return ans;
		}
		
		inline int read(int mod)
		{
			int x = 0;
			bool g = false;
			char c = getchar();
			while (c < '0' || c > '9')
			c = getchar();
			while (c >= '0' && c <= '9')
			{
				x = (x << 3) + (x << 1) + (c ^ '0');
				if (x >= mod)
				x %= mod, g = true;
				c = getchar();
			}
			if (g)
			return (x + mod);
			else
			return x;
		}
		main()
		{
			int a, b, m;
			cin >> a >> m;
			int ph = phi(m);
			int ans = 0;
			ans = qpow(a, read(ph), m);
			cout << ans << endl;
			system("pause");
		}
	\end{lstlisting}
	\subsection{逆元}
	\subsubsection{费马小定理}
	要求$p$是质数
	\begin{lstlisting}[caption={}]
		inline ll qpow(ll a, ll n, ll p)// 快速幂
		{
			ll ans = 1;
			while (n)
			{
				if (n & 1)
				ans = ans % p * a % p;
				a = a % p * a % p;
				n >>= 1;
			}
			return ans;
		}
		inline ll inv(ll a, ll p)
		{
			return qpow(a, p - 2, p);
		}
	\end{lstlisting}
	\subsubsection{扩展欧几里得}
	\begin{lstlisting}[caption={}]
		ll exgcd(ll a, ll b, ll &x, ll &y)// 拓欧
		{
			if (b == 0)
			{
				x = 1;
				y = 0;
				return a;
			}
			ll d = exgcd(b, a % b, y, x);
			y -= (a / b) * x;
			return d;
		}
		ll inv(ll a, ll p)
		{
			ll x, y;
			if (exgcd(a, p, x, y) != 1) // 无解的情形
			return -1;
			return (x % p + p) % p;
		}
	\end{lstlisting}
	\subsubsection{线性递推}
	\begin{lstlisting}[caption={}]
		inv[1] = 1;
		for (int i = 2; i <= n; ++i) {
			inv[i] = (long long)(p - p / i) * inv[p % i] % p;
		}
	\end{lstlisting}
	\subsection{大步小步算法}
	解$a^x \equiv b (\bmod m)$, 其中$a$, $b$, $m$互质。
	\begin{lstlisting}[caption={}]
		ll BSGS(ll a, ll b, ll m)
		{
			static unordered_map<ll, ll> hs;
			hs.clear();
			ll cur = 1, t = sqrt(m) + 1;
			for (int B = 1; B <= t; ++B)
			{
				(cur *= a) %= m;
				hs[b * cur % m] = B; // 哈希表中存B的值
			}
			ll now = cur; // 此时cur = a^t
			for (int A = 1; A <= t; ++A)
			{
				auto it = hs.find(now);
				if (it != hs.end())
				return A * t - it->second;
				(now *= cur) %= m;
			}
			return -1; // 没有找到,无解
		}
	\end{lstlisting}
	还有扩展大步小步,可以解$a$和$m$不互质的式子
	\begin{lstlisting}[caption={}]
		// 修改版的BSGS,额外带一个系数
		ll BSGS(ll a, ll b, ll m, ll k = 1)
		{
			static unordered_map<ll, ll> hs;
			hs.clear();
			ll cur = 1, t = sqrt(m) + 1;
			for (int B = 1; B <= t; ++B)
			{
				(cur *= a) %= m;
				hs[b * cur % m] = B; // 哈希表中存B的值
			}
			ll now = cur * k % m;
			for (int A = 1; A <= t; ++A)
			{
				auto it = hs.find(now);
				if (it != hs.end()) return A * t - it->second;
				(now *= cur) %= m;
			}
			return -INF; // 这里因为要多次加1,要返回更小的负数
		}
		ll exBSGS(ll a, ll b, ll m, ll k = 1)
		{
			ll A = a %= m, B = b %= m, M = m;
			if (b == 1) return 0;
			ll cur = 1 % m;
			for (int i = 0;; i++)
			{
				if (cur == B) return i;
				cur = cur * A % M;
				ll d = gcd(a, m);
				if (b % d) return -INF;
				if (d == 1) return BSGS(a, b, m, k * a % m) + i + 1;
				k = k * a / d % m, b /= d, m /= d; // 相当于在递归求解exBSGS(a, b / d, m / d, k * a / d % m)
			}
		}
	\end{lstlisting}
	\subsection{米勒拉宾素数检验}
	\begin{lstlisting}
		bool is_prime(ll x)
		{
			if (x < 3) // 特判1,2
			return x == 2;
			if (x % 2 == 0) // 特判偶数
			return false;
			ll A[] = {2, 325, 9375, 28178, 450775, 9780504, 1795265022}, d = x - 1, r = 0;
			while (d % 2 == 0) // 算出d, r 
			d /= 2, ++r;
			// 或:r = __builtin_ctz(d), d >>= r;
			for (auto a : A)
			{
				ll v = qpow(a, d, x); // a^d
				// 如果a^d≡0,说明是a是x的倍数;如果a^d≡1或-1,说明这串数接下来一定都是1,不用继续计算
				if (v <= 1 || v == x - 1) 
				continue;
				for (int i = 0; i < r; ++i)
				{
					v = (__int128)v * v % x; // 同样使用__int128过渡
				
					if (v == x - 1 && i != r - 1) // 得到-1,说明接下来都是1,可以退出了
					{
						v = 1;
						break;
					}
					// 在中途而非开头得到1,却没有经过-1,说明存在其他数字y≠-1满足y^2≡1,则x一定不是奇素数
					if (v == 1)  
					return false;
				}
				if (v != 1) // 查看是不是以1结尾
				return false;
			}
			return true;
		}
	\end{lstlisting}
	\subsection{rho}
		\begin{lstlisting}[caption={}]
			ll Pollard_Rho(ll N)
			{
				if (N == 4)
				return 2;
				if (is_prime(N))
				return N;
				while (1)
				{
					ll c = randint(1, N - 1);
					auto f = [=](ll x) { return ((lll)x * x + c) % N; };
					ll t = 0, r = 0, p = 1, q;
					do
					{
						for (int i = 0; i < 128; ++i) // 令固定距离C=128
						{
							t = f(t), r = f(f(r));
							if (t == r || (q = (lll)p * abs(t - r) % N) == 0) // 如果发现环,或者积即将为0,退出
							break;
							p = q;
						}
						ll d = gcd(p, N);
						if (d > 1)
						return d;
					} while (t != r);
				}
			}
		\end{lstlisting}
		可以求所有素因数或者最大质因数
		\begin{lstlisting}[caption={}]
			void find_prime_factor(int n) {
				if (n == 1 || vis.count(n)) {
					return;
				}
				
				if (is_prime(n)) {
					prime.push_back(n);
					return;
				}
				
				vis[n] = 1;
				int x = Pollard_Rho(n);
				find_prime_factor(x);
				find_prime_factor(n / x);
			}
			unordered_map<ll, ll> um;
			ll max_prime_factor(ll x)
			{
				if (um.count(x))
				return um[x];
				ll fac = Pollard_Rho(x);
				if (fac == 1)
				um[x] = x;
				else
				um[x] = max(max_prime_factor(fac), max_prime_factor(x / fac));
				return um[x];
			}
		\end{lstlisting}
	\subsection{康托展开}
	\begin{lstlisting}[caption={}]
		#include <bits/stdc++.h>
		using namespace std;
		
		#define int long long
		const int mod = 998244353;
		const int N = 1e6 + 1e2;
		
		int tree[N], fac[N];
		
		int lowbit(int x) {
			return x & -x;
		}
		void update(int x, int k) {
			while (x < N - 100) {
				tree[x] += k;
				x += lowbit(x);
			}
		}
		int query(int x) {
			int ans = 0;
			while (x) {
				ans += tree[x];
				x -= lowbit(x);
			}
			return ans;
		}
		main() {
			int n;
			cin >> n;
			fac[0] = 1;
			for (int i = 1; i <= n; ++i) {
				fac[i] = fac[i - 1] * i % mod;
				update(i, 1);
			}
			int ans = 0;
			for (int i = 1; i <= n; ++i) {
				int x;
				cin >> x;
				ans = (ans + query(x - 1) * fac[n - i] % mod) % mod;
				update(x, -1);
			}
			cout << ans + 1 << endl;
			system("pause");
		}
	\end{lstlisting}
	\subsection{中国剩余定理(CRT)}
	
	解
	$$\begin{cases}
		\quad x \equiv a_{1} (\bmod p_{1})\\
		\quad x \equiv a_{2} (\bmod p_{2})\\
		\ldots\ldots\\
		\quad x \equiv a_{k} (\bmod p_{k})\\
	\end{cases}$$
	其中$m_{1}, m_{2}, \ldots, m_{k}$是两两互质的整数,中国剩余定理可以给出$x$的最小非负整数解。
	\begin{lstlisting}[caption={}]
		#include <bits/stdc++.h>
		using namespace std;
		
		typedef long long LL;
		
		LL exgcd(LL a, LL b, LL &x, LL &y) {
			if (b == 0) {
				x = 1;
				y = 0;
				return a;
			}
			int d = exgcd(b, a % b, y, x);
			y -= (a / b) * x;
			return d;
		}
		LL CRT(int k, LL *a, LL *p) {
			LL mul = 1;
			for (int i = 0; i < k; ++i) {
				mul *= p[i];
			}
			LL ans = 0;
			for (int i = 0; i < k; ++i) {
				LL m = mul / p[i], x, y;
				exgcd(m, p[i], x, y);
				ans = (ans + a[i] * m * x % mul) % mul;
			}
			return (ans % mul + mul) % mul;
		}
	\end{lstlisting}
	\subsection{扩展中国剩余定理(exCRT)}
	同中国剩余定理,但无任何限制
	
	(考不到吧)
	\begin{lstlisting}[caption={}]
		#include <bits/stdc++.h>
		using namespace std;
		
		#define int long long
		const int N = 1e5 + 1e3;
		
		int mod[N], yu[N];
		
		int lcm(int a, int b) {
			return a / __gcd(a, b) * b;
		}
		
		int qmul(int n, int k, int mod) {
			int ans = 0;
			while (k) {
				if (k % 2) {
					ans = (ans + n) % mod;
				}
				k /= 2;
				n = (n + n) % mod;
			}
			return ans % mod;
		}
		
		int exgcd(int a, int b, int &x, int &y) {
			if (!b) {
				x = 1;
				y = 0;
				return a;
			}
			int d = exgcd(b, a % b, y, x);
			y -= (a / b) * x;
			return d;
		}
		
		main() {
			int n;
			cin >> n;
			for (int i = 0; i < n; ++i) {
				cin >> mod[i] >> yu[i];
			}
			int ans = yu[0], M = mod[0], t, y;
			for (int i = 1; i < n; ++i) {
				int mi = mod[i];
				int res = ((yu[i] - ans) % mi + mi) % mi;
				int d = exgcd(M, mi, t, y);
				if (res % d) {
					cout << "-1\n";
					exit(0);
				}
				t = qmul(t, res / d, mi);
				ans += t * M;
				M = lcm(M, mi);
				ans = (ans % M + M) % M;
			}
			cout << ans << endl;
			system("pause");
		}
	\end{lstlisting}
	\subsection{高斯消元}
	\begin{lstlisting}[caption={}]
		#include <bits/stdc++.h>
		using namespace std;
		
		vector<valarray<double>> a;
		const double eps = 1e-10;
		int n;
		
		template <typename it, typename unf, typename bif>
		void line_elimin(it &arr, const unf &ck, const bif &op)
		{
			for (auto &i : arr)
			{
				if (!ck(i))
				continue;
				for (auto &j : arr)
				if (&i != &j)
				op(i, j);
			}
		}
		int main()
		{
			// int n;
			cin >> n;
			a = vector<valarray<double>>(n, valarray<double>(n + 1));
			for (int i = 0; i < n; ++i)
			{
				for (int j = 0; j < n + 1; ++j)
				{
					cin >> a[i][j];
				}
			}
			auto ck = [](const valarray<double> &x)
			{
				for (int i = 0; i < x.size() - 1; ++i)
				{
					if (abs(x[i]) > eps)
					{
						return 1;
					}
				}
				return 0;
			};
			auto op = [](const valarray<double> &a, valarray<double> &b)
			{
				int i = 0;
				while (abs(a[i]) < eps)
				{
					++i;
				}
				valarray<double> x = b[i] * a / a[i];
				b -= x;
			};
			line_elimin(a, ck, op);
			for (int i = 0; i < n; ++i)
			{
				if (!ck(a[i]))
				{
					if (abs(a[i][a[i].size() - 1]) > eps) {
						puts("-1");
						return 0;
					}
				}
			}
			for (int i = 0; i < n; ++i)
			{
				if (!ck(a[i]))
				{
					if (abs(a[i][a[i].size() - 1]) <= eps)
					{
						puts("0");
						return 0;
					}
				}
			}
			for (int i = 0; i < n; ++i)
			{
				int j = 0;
				for (; j < n; ++j)
				{
					if (abs(a[j][i]) > eps)
					{
						break;
					}
				}
				cout << "x" << i + 1 << "=" << fixed << setprecision(2) << a[j][n] / a[j][i] << endl;
			}
		}
	\end{lstlisting}
	\subsection{线性基}
	\begin{lstlisting}[caption={}]
		bitset<N> B[N];
		void insert(bitset<N> x)
		{
			for (int i = 0; i < N; ++i)
			if (x[i])
			x ^= B[i];
			if (x == 0) return;
			int msb = N - 1;
			while (msb && x[msb] == 0) msb--;
			B[msb] = x;
			for (int i = msb + 1; i < N; ++i)
			if (B[i][msb])
			B[i] ^= x;
		}
	\end{lstlisting}
	\subsection{杜教筛}
	\begin{lstlisting}[caption={}]
		#define int long long
		
		const int N = 1e6 + 1e2;
		using i64 = long long;
		
		int prime[N];
		vector<int> primeset;
		i64 phi[N], sumphi[N];
		i64 mu[N], summu[N];
		void init() {
			phi[1] = mu[1] = 1;
			for (int i = 2; i <= N - 100; ++i) {
				if (!prime[i]) {
					phi[i] = i - 1;
					mu[i] = -1;
					primeset.push_back(i);
				}
				for (int u : primeset) {
					if (u * i > N - 100) {
						break;
					}
					prime[i * u] = 1;
					if (i % u == 0) {
						phi[i * u] = phi[i] * u;
						break;
					} else {
						mu[i * u] = mu[i] * mu[u];
						phi[i * u] = phi[i] * phi[u];
					}
				}
			}
			for (int i = 1; i <= N - 100; ++i) {
				//cout << phi[i] << endl;
				summu[i] = summu[i - 1] + mu[i];
				sumphi[i] = sumphi[i - 1] + phi[i];
			}
		}
		namespace SumPhi {
			unordered_map<int, i64> mp;
			i64 f(int n) {
				i64 ans = n * (n + 1) / 2;
				if (n <= N - 100) {
					return sumphi[n];
				}
				if (mp.count(n)) {
					return mp[n];
				}
				int l = 2, r = n;
				while (l <= n) {
					r = n / (n / l);
					ans -= (r - l + 1) * f(n / l);
					l = r + 1;
				}
				return mp[n] = ans;
			}
		}
		unordered_map<int, i64> mp;
		i64 f(int n) {
			i64 ans = 1;
			if (n <= N - 100) {
				return summu[n];
			}
			if (mp.count(n)) {
				return mp[n];
			}
			int l = 2, r = n;
			while (l <= n) {
				r = n / (n / l);
				ans -= (r - l + 1) * f(n / l);
				l = r + 1;
			}
			return mp[n] = ans;
		}
	\end{lstlisting}
	\subsection{一些式子}
	$\sum_{i = 1}^{n} i ^ x = \frac{C(x + 1, n + 1)}{(n + 1)!} + \frac{C(x, n + )}{分母}$
	
	
	\section{多项式}
	\subsection{fft(快速傅里叶变换)}
	\begin{lstlisting}[caption={}]
		#include <bits/stdc++.h>
		using namespace std;
		
		const int N = 1e6 + 1e2;
		
		typedef complex<double> comp;
		comp a[N * 3], b[N * 3], tmp[N * 3], ans[N * 3];
		int rev[N * 3];
		const comp I(0, 1);
		const double PI = M_PI;
		void fft(comp *f, int len, int op = 1) {
			int bit = __lg(len);
			for (int i = 0; i < len; ++i) {
				rev[i] = (rev[i >> 1] >> 1) | ((i & 1) << (bit - 1));
				if (i < rev[i]) {
					swap(f[i], f[rev[i]]);
				}
			}
			for (int l = 1; l < len; l <<= 1) {
				comp step = exp(PI / l * op * I);
				for (int i = 0; i < len; i += 2 * l) {
					comp cur(1, 0);
					for (int k = i; k < i + l; ++k) {
						comp g = f[k], h = f[k + l] * cur;
						f[k] = g + h, f[k + l] = g - h;
						cur *= step;
					}
				}
			}
		}
		int main() {
			int n, m;
			cin >> n >> m;
			int len = 1 << __lg(n + m + 1) + 1;
			for (int i = 0; i <= n; ++i) {
				cin >> a[i];
			}
			for (int i = 0; i <= m; ++i) {
				cin >> b[i];
			}
			fft(a, len);
			fft(b, len);
			for (int i = 0; i <= len; ++i) {
				ans[i] = a[i] * b[i];
			}
			fft(ans, len, -1);
			for (int i = 0; i <= n + m; ++i) {
				cout << int(ans[i].real() / len + 0.1) << ' ';
			}
			cout << endl;
			system("pause");
		}
	\end{lstlisting}
	\section{计算几何}
	\subsection{极角排序}
	\begin{lstlisting}[caption={}]
		const double eps = 0;
		#define x first
		#define y second
		typedef pair<double, double> p;
		using i64 = long long;
		vector<p> v;
		double f(double x) {
			if (abs(x) <= eps) {
				return 0;
			}
			return x;
		}
		i64 cross(p &A, p &B) {
			return A.x * B.y - A.y * B.x;
		}
		i64 dul(p &A, p &B) {
			return A.x * B.x + A.y * B.y;
		}
		int f(p A) {
			if (A.y > eps) {
				if (A.x > eps) {
					return 1;
				}
				return 2;
			} else {
				if (A.x > eps) {
					return 4;
				}
				return 3;
			}
		}
		bool cmp(p &A, p &B) {
			return f(A) < f(B) || (f(A) == f(B) && cross(A, B) > 0);
		}
		void qsort() {
			sort(v.begin(), v.end(), cmp);
		}
		int len(p &A) {
			return A.x * A.x + A.y * A.y;
		}
		double cos(p &A, p &B) {
			return dul(A, B) / len(A) / len(B);
		}
	\end{lstlisting}
	\subsection{求凸包}
	\begin{lstlisting}[caption={}]
		p operator- (p A, p B) {
			return {A.x - B.x, A.y - B.y};
		}
		int cross(p A, p B) {
			return A.x * B.y - A.y * B.x;
		}
		void f(vector<p> v, vector<p> &p) {
			for (int i = 0; i < v.size(); ++i) {
				while (p.size() >= 2 && cross(v[i] - p.back(), p[p.size() - 2] - p.back()) < 0) {
					p.pop_back();
				}
				p.push_back(v[i]);
			}
		}
	\end{lstlisting}
	\subsection{求凸包直径}
	\begin{lstlisting}[caption={}]
		const double pi = 3.141592653589793;
		const double eps=1e-12;
		int dcmp(double x){
			return (fabs(x)<=eps)?0:(x<0?-1:1);
		}
		struct Point{
			double x,y;
			Point(double X=0,double Y=0){x=X,y=Y;}
		};
		struct Vector{
			double x,y;
			Vector(double X=0,double Y=0){x=X,y=Y;}
		};
		inline Vector operator-(Point x,Point y){// 点-点=向量
			return Vector(x.x-y.x,x.y-y.y);
		}
		inline double cross(Vector x,Vector y){ // 向量叉积
			return x.x*y.y-x.y*y.x;
		}
		inline double operator*(Vector x,Vector y){ // 向量叉积
			return cross(x,y);
		}
		inline double len(Vector x){ // 向量模长
			return sqrt(x.x*x.x+x.y*x.y);
		}
		
		int stk[N];
		bool used[N];
		vector<Point> ConvexHull(Point* poly, int n){ // Andrew算法求凸包
			int top=0;
			sort(poly+1,poly+n+1,[&](Point x,Point y){
				return (x.x==y.x)?(x.y<y.y):(x.x<y.x);
			});
			for (int i = 0; i <= n; ++i) {
				used[i] = 0;
			}
			stk[++top]=1;
			for(int i=2;i<=n;i++){
				while(top>1&&dcmp((poly[stk[top]]-poly[stk[top-1]])*(poly[i]-poly[stk[top]]))<=0){
					used[stk[top--]]=0;
				}
				used[i]=1;
				stk[++top]=i;
			}
			int tmp=top;
			for(int i=n-1;i;i--){
				if(used[i]) continue;
				while(top>tmp&&dcmp((poly[stk[top]]-poly[stk[top-1]])*(poly[i]-poly[stk[top]]))<=0){
					used[stk[top--]]=0;
				}
				used[i]=1;
				stk[++top]=i;
			}
			vector<Point> a;
			for(int i=1;i<=top;i++){
				a.push_back(poly[stk[i]]);
			}
			return a;
		}
		
		struct Line{
			Point x;Vector y;
			Line(Point X,Vector Y){x=X,y=Y;}
			Line(Point X,Point Y){x=X,y=Y-X;}
		};
		
		inline double DistanceToLine(Point P,Line x){// 点到直线的距离
			Vector v1=x.y, v2=P-x.x;
			return fabs(cross(v1,v2))/len(v1);
		}
		
		double RoatingCalipers(vector<Point> poly){// 旋转卡壳
			if(poly.size()==3) return len(poly[1]-poly[0]);
			int cur=0;
			double ans=0;
			for(int i=0;i<poly.size()-1;i++){
				Line line(poly[i],poly[i+1]);
				while(DistanceToLine(poly[cur], line) <= DistanceToLine(poly[(cur+1)%poly.size()], line)){
					cur=(cur+1)%poly.size();
				}
				ans=max(ans, max(len(poly[i]-poly[cur]), len(poly[i+1]-poly[cur])));
			}
			return ans;
		}
	\end{lstlisting}
	\section{字符串}
	\subsection{KMP}
	\subsubsection{MP}
	\begin{lstlisting}[caption={}]
		vector<int> pmt(p.length() + 1, 0);
		for (int i = 1, j = 0; i < p.length(); ++i) {
			while (j && p[i] != p[j]) {
				j = pmt[j - 1];
			}
			if (p[i] == p[j]) {
				++j;
			}
			pmt[i] = j;
		}
		for (int i = 0, j = 0; i < s.length(); ++i) {
			while (j && s[i] != p[j]) {
				j = pmt[j - 1];
			}
			if (s[i] == p[j]) {
				++j;
			}
			if (j == p.length()) {
				cout << i - j + 2 << endl;
				j = pmt[j - 1];
			}
		}
	\end{lstlisting}
	\subsubsection{KMP}
	pmt含义改变了!
	\begin{lstlisting}[caption={}]
		vector<int> pmt(p.length() + 1, 0);
		for (int i = 1, j = 0; i < p.length(); ++i) {
			while (j && p[i] != p[j]) {
				j = pmt[j - 1];
			}
			if (p[i] == p[j]) {
				++j;
				if (p[i + 1] == p[j + 1]) {
					pmt[i] = pmt[j];
					continue;
				}
			}
			pmt[i] = j;
		}
		for (int i = 0, j = 0; i < s.length(); ++i) {
			while (j && s[i] != p[j]) {
				j = pmt[j - 1];
			}
			if (s[i] == p[j]) {
				++j;
			}
			if (j == p.length()) {
				cout << i - j + 2 << endl;
				j = pmt[j - 1];
			}
		}
	\end{lstlisting}
	\subsection{exKMP}
	\begin{lstlisting}[caption={}]
		vector<int> get_Z(const string &s) {
			vector<int> z(s.length(), 0);
			z[0] = s.length();
			int l = 0, r = 0;
			for (int i = 1; i < s.length(); ++i) {
				if (i > r || i + z[i - l] > r) {
					z[i] = max(0, r - i + 1);
					while (i + z[i] < s.length() && s[z[i]] == s[i + z[i]]) {
						z[i]++;
					}
					l = i, r = i + z[i] - 1;
					continue;
				}
				z[i] = z[i - l];
			}
			return z;
		}
	\end{lstlisting}
\end{document}